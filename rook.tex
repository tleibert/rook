\documentclass[10pt]{article}

\begin{document}

\title{Rook Rules}
\author{Savannah Aker
    \and
    Trevor Leibert}
\date{\today}

\maketitle
\pagebreak

\section{Introduction}

This is the game of Rook, as played by the ancestors of Savannah Aker.
This is a game for four players, played in teams of two.
Players on the same team are seated across from each other.

\section{Setup}

The game is played with a standard Rook deck, with the following modifications:

\begin{enumerate}
    \item All cards of value 4 and below are removed from the deck.
          This should leave 10 cards of each color or suit.
    \item The black and red ones should be re-added to the deck.
          These two cards, together with the Rook card, are considered the three ``high-value'' cards.
\end{enumerate}

At the start of a game, the deck should be shuffled

\section{Rules}

Each game is played as a series of rounds, with the overall goal of reaching 500 points.
Each round is split into nine tricks, or ``hands''.

\subsection{The Round}

Each round begins with the dealer dealing out nine cards to each player.
Seven cards should remain, and be placed in the center of the table, with the top card on the pile face-up.
This grouping of cards is known as the ``kitty.''
Each player should look at their cards, but not show them to anyone else.
After each player looks at their cards, the bidding begins.

\subsubsection{Bidding}

\subsection{The Trick}

The first trick in a round is played by the winner of the kitty.

\subsection{Card Values}

The sum of available points per round is 175.
Card point values are listed in descending order:
\begin{itemize}
    \item Black One: 30 points
    \item Red One: 25 points
    \item Rook: 20 points
    \item 14's and 10's: 10 points each
    \item 5's: 5 points each
\end{itemize}


\end{document}
