\documentclass[10pt]{article}

\begin{document}

\title{Rook Rules}
\author{Savannah Aker
    \and
    Trevor Leibert}
\date{\today}

\maketitle
\pagebreak

\section{Introduction}

This is the game of Rook, as played by the ancestors of Savannah Aker.
This is a game for four players, played in teams of two.
Players on the same team are seated across from each other.

\section{Setup}

The game is played with a standard Rook deck, with the following modifications:

\begin{enumerate}
    \item All cards of value 4 and below are removed from the deck.
          This should leave 10 cards of each color or suit.
    \item The black and red ones should be re-added to the deck.
          These two cards, together with the Rook card, are considered the three ``high-value'' cards.
\end{enumerate}

At the start of a game, the deck should be shuffled

\section{Rules}

Each game is played as a series of rounds, with the overall goal of reaching 500 points.
Each round is split into nine tricks, or ``hands.''

\subsection{The Round}

Each round begins with the dealer dealing out nine cards to each player.
Seven cards should remain, and be placed in the center of the table, with the top card on the pile face-up.
This grouping of cards is known as the ``kitty.''
Each player should look at their cards, but not show them to anyone else.
After each player looks at their cards, the bidding begins.

\subsubsection{Bidding}

Bidding begins with the player to the left of the dealer.
Each player in turn must either bid or pass.
Bids are made by saying the number of points that the player thinks their team can win in the round, out of the total 175 points available.
The player who bids the highest number of points wins the kitty.
Bidding must continue until there is a winning bid.

The winning player must then take the kitty, and assemble their ideal nine-card hand from their original nine cards and the seven cards in the kitty.
Seven cards will be left over from this process, and should be placed aside face-down. These cards go to the winner of the last trick in the round.
The winning team must reach their bid in points for the round. If they do not, their team's score will be decreased (``go set'') by the number of points they bid, and no points from the round will be added to their score.
The team that didn't win the kitty has no liability, and will simply add their points from the round to their total score.

If a player thinks they can win all 175 points, they may ``shoot the moon,'' which instantly wins the bidding and the kitty:
\begin{itemize}
    \item If they successfully win all 175 points, their team's score will be increased by an extra 125 points, for a total of 300 points for the round.
    \item If they fail to win all 175 points, their team's score will be decreased by 300 points, and no points from the round will be added to their score.
\end{itemize}

\subsubsection{Trumps}

The winner of the kitty chooses a color to be the trump color for the round.
The three high-value cards assume this color as their color for the duration of the round.

\subsection{The Trick}

The first trick in a round is lead by the winner of the kitty.
The winner of the trick leads the next trick, and so on.

\subsection{Card Values}

The sum of available points per round is 175.
Card point values are listed in descending order:
\begin{itemize}
    \item Black One: 30 points
    \item Red One: 25 points
    \item Rook: 20 points
    \item 14's and 10's: 10 points each
    \item 5's: 5 points each
\end{itemize}


\end{document}
